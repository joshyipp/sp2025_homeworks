\documentclass[10pt]{article}
\usepackage{amsmath, amssymb}

\begin{document}

\begin{center}
    \LARGE {Problem Set 2 – Shallow and Deep Networks} \\[1em]
    \Large{DS542 – DL4DS} \\[0.5em]
    \large Spring, 2025
\end{center}

\vspace{2em}

\noindent\textbf{Note:} Refer to the equations in the \textit{Understanding Deep Learning} textbook to solve the following problems.

\vspace{2em}

\section*{Problem 3.2}
For each of the four linear regions in Figure 3.3j, indicate which hidden units are inactive and which are active (i.e., which do and do not clip their inputs).
\begin{enumerate}
    \item First linear region:
    \item Second linear region:
    \item Third linear region: 
    \item Fourth linear region: 
\end{enumerate}
\vspace{2em}

\section*{Problem 3.5}

Prove that the following property holds for $\alpha \in \mathbb{R}^+$:
\[
\text{ReLU}[\alpha \cdot z] = \alpha \cdot \text{ReLU}[z].
\]
This is known as the non-negative homogeneity property of the ReLU function.

\vspace{2em}

\section*{Problem 4.6}

Consider a network with $D_i = 1$ input, $D_o = 1$ output, $K = 10$ layers, and $D = 10$ hidden units in each. Would the number of weights increase more -- if we increased the depth by one or the width by one? Provide your reasoning.


\end{document}
